\chapter{Graphs and Motifs} \label{chap:graphs}

We describe our graph-theoretic framework for network analysis and give matrix-based formulae for motif adjacency matrices (MAMs).
In Section~\ref{sec:graphs_graph_definitions} we outline basic concepts relating to graphs and motifs.
In Section~\ref{sec:graphs_adj_and_ind_matrices} we define the adjacency and indicator matrices of a graph.
In Section~\ref{sec:graphs_motif_adj_matrices} we introduce MAMs and present the main results of this chapter, Proposition~\ref{prop:motif_adj_matrix_formula} and Proposition~\ref{prop:motif_adj_matrix_computation}.








\section{Graph definitions} \label{sec:graphs_graph_definitions}

Graph notation is notoriously inconsistent in the literature \cite{intro_to_graph_theory}, so we begin by giving all of the relevant notation and definitions.

\begin{definition}[Graphs]
A \emph{graph} is a triple $\ca{G} = (\ca{V,E},W)$ where $\ca{V}$ is the \emph{vertex set}, $\ca{E} \subseteq \left\{ (i,j) : i,j \in \ca{V}, i \neq j \right\}$ is the \emph{edge set} and $W\colon \ca{E} \to (0,\infty)$ is the \emph{weight map}.
\end{definition}

\begin{remark}
We consider weighted directed graphs without self-loops or multiple edges. We can extend to undirected graphs by replacing undirected edges with bidirectional edges. Where it is not relevant, we may sometimes omit the weight map $W$.
\end{remark}

\begin{definition}[Underlying edges]
Let $\ca{G} = (\ca{V,E})$ be a graph. Its \emph{underlying edges} are $\bar{\ca{E}} \vcentcolon = \big\{ \{i,j\} : (i,j) \in \ca{E} \big\}$.
\end{definition}

\begin{definition}[Subgraphs]
A graph $\ca{G'} = (\ca{V',E'})$ is a \emph{subgraph} of a graph $\ca{G} = (\ca{V,E})$ (write $\ca{G'} \leq \ca{G}$) if $\ca{V'} \subseteq \ca{V}$ and $\ca{E'} \subseteq \ca{E}$. It is an \emph{induced subgraph} (write $\ca{G'} < \ca{G}$) if further $\ca{E'} = \ca{E} \cap ( \ca{V'} \times \ca{V'} )$.
\end{definition}

\begin{definition}[Connected components]
Let $\ca{G} = (\ca{V,E})$ be a graph. The \emph{connected components} of $\ca{G}$ are the partition $\ca{C}$ generated by the transitive closure of the relation $\sim$ on $\ca{V}$ defined by $i \sim j \iff \{i,j\} \in \bar{\ca{E}}$. We say $\ca{G}$ is (weakly) \emph{connected} if $|\ca{C}| = 1$.
\end{definition}

\begin{definition}[Graph isomorphisms]
A graph $\ca{G'} = (\ca{V',E'})$ is \emph{isomorphic} to a graph $\ca{G} = (\ca{V,E})$ (write $\ca{G'} \cong \ca{G}$) if there exists a bijection $\phi\colon \ca{V'} \rightarrow \ca{V}$ with $(u,v) \in \ca{E'} \iff \big(\phi(u), \phi(v) \big) \in \ca{E}$.
An isomorphism from a graph to itself is called an \emph{automorphism}.
\end{definition}

\begin{definition}[Motifs and anchor sets]
A \emph{motif} is a pair $(\ca{M,A})$ where $\ca{M} = (\ca{V_M,E_M})$ is a connected graph with $\ca{V_M} = \{ 1, \ldots, m \}$ for some small $m \geq 2$, and $\ca{A} \subseteq \ca{V_M}$ with $|\ca{A}| \geq 2$ is an \emph{anchor set}. If $\ca{A} \neq \ca{V_M}$ we say the motif is \emph{anchored}, and if $\ca{A=V_M}$ we say it is \emph{simple}.
\end{definition}

\begin{remark}
Anchor sets~\cite{benson2016higher} specify which r\^oles vertices play in the motif, and are crucial for defining the collider and expander motifs given in Section~\ref{sec:coll_expa}. When an anchor set is not given, it is assumed that the motif is simple. Figure~\ref{fig:motif_definitions_directed} shows all simple motifs (up to isomorphism) on at most three vertices. 
\end{remark}

\begin{definition}[Instances]
Let $\ca{G}$ be a graph and $(\ca{M,A})$ a motif. We say that $\ca{H}$ is a \emph{functional instance} of $\ca{M}$ in $\ca{G}$ if $\ca{M} \cong \ca{H} \leq \ca{G}$. We say that $\ca{H}$ is a \emph{structural instance} of $\ca{M}$ in $\ca{G}$ if $\ca{M} \cong \ca{H} < \ca{G}$.
\end{definition}

\begin{definition}[Anchored pairs]
Let $\ca{G}$ be a graph and $(\ca{M,A})$ a motif. Suppose $\ca{H}$ is an instance of $\ca{M}$ in $\ca{G}$. Define the \emph{anchored pairs of the instance} $\ca{H}$ as
$$ \ca{A(H)} \vcentcolon = \big\{ \{\phi(i),\phi(j)\} : i,j \in \ca{A}, \ i \neq j, \ \phi \textrm{ is an isomorphism from } \ca{M} \textrm{ to } \ca{H} \big\}\,.$$

\end{definition}

\begin{remark}
Example~\ref{ex:instances} demonstrates functional and structural instances. Note that $\{i,j\} \in \ca{A(H)}$ if and only if $\ca{H}$ appears in $\ca{G}$ as an instance of $\ca{M}$ with $i \neq j$ co-appearing in the image of $\ca{A}$ under isomorphism. The motivation for this is that clustering methods should avoid separating vertices which appear as an anchored pair.
\end{remark}

\begin{figure}[H]
	\centering
	\includegraphics[scale=0.7,draft=false]{../tikz/motif_definitions_directed/motif_definitions_directed.pdf}
	\caption{All simple motifs on at most three vertices}
	\label{fig:motif_definitions_directed}
\end{figure}











\section{Adjacency and indicator matrices} \label{sec:graphs_adj_and_ind_matrices}

Adjacency matrices provide a useful data structure for representing graphs and have many uses in calculating graph properties \cite{bapat2010graphs}.  We define several variants of the adjacency matrix, which appear in Proposition~\ref{prop:motif_adj_matrix_formula} and Table~\ref{tab:motif_adj_mat_table}.

\begin{definition}[Adjacency matrices]
Let $\ca{G} = (\ca{V,E},W)$ be a graph with vertex set $\ca{V} = \{1, \ldots, n \}$. The \emph{adjacency matrix, single-edge adjacency matrix} and \emph{double-edge adjacency matrix} of $\ca{G}$ are respectively the $n \times n$ matrices
\begin{align*}
	G_{ij} &\vcentcolon= W((i,j)) \ \bb{I} \{ (i,j) \in \ca{E} \}\,, \\
	(G_\mathrm{s})_{ij} &\vcentcolon= W((i,j)) \ \bb{I} \{ (i,j) \in \ca{E} \textrm{ and } (j,i) \notin \ca{E} \}\,, \\
	(G_\mathrm{d})_{ij} &\vcentcolon= \big( W((i,j)) + W((j,i)) \big) \ \bb{I} \{ (i,j) \in \ca{E} \textrm{ and } (j,i) \in \ca{E} \}\,. 
\end{align*}
\end{definition}

\begin{definition}[Indicator matrices]
Let $\ca{G} = (\ca{V,E},W)$ be a graph with vertex set $\ca{V} = \{1, \ldots, n \}$. The \emph{indicator matrix, single-edge indicator matrix, double-edge indicator matrix, missing-edge indicator matrix} and \emph{vertex-distinct indicator matrix} of $\ca{G}$ are respectively the $n \times n$ matrices
\begin{align*}
	J_{ij} &\vcentcolon= \bb{I} \{ (i,j) \in \ca{E} \}\,, \\
	(J_\mathrm{s})_{ij} &\vcentcolon= \bb{I} \{ (i,j) \in \ca{E} \textrm{ and } (j,i) \notin \ca{E} \}\,, \\
	(J_\mathrm{d})_{ij} &\vcentcolon= \bb{I} \{ (i,j) \in \ca{E} \textrm{ and } (j,i) \in \ca{E} \}\,, \\
	(J_0)_{ij} &\vcentcolon= \bb{I} \{ (i,j) \notin \ca{E} \textrm{ and } (j,i) \notin \ca{E} \textrm{ and } i \neq j \}\,, \\
	(J_\mathrm{n})_{ij} &\vcentcolon= \bb{I} \{ i \neq j \}\,.
\end{align*}
\end{definition}











\section{Motif adjacency matrices} \label{sec:graphs_motif_adj_matrices}

The central object in motif-based spectral clustering is the \emph{motif adjacency matrix} (MAM) \cite{benson2016higher}, which serves as a similarity matrix for spectral clustering (Chapter~\ref{chap:spectral}).
We provide here our main results: Proposition~\ref{prop:motif_adj_matrix_formula} gives a computationally useful formula for MAMs, and Proposition~\ref{prop:motif_adj_matrix_computation} gives a complexity analysis of this formula.









\pagebreak

\subsection{Definitions}

\begin{definition}[Motif adjacency matrices] \label{def:motif_adj_matrices}
%
Let $\ca{G} = (\ca{V,E},W)$ be a graph with $n$ vertices and let $\ca{(M,A)}$ be a motif. The \emph{functional} and \emph{structural motif adjacency matrices} (MAMs) of $\ca{(M,A)}$ in $\ca{G}$ are respectively the $n \times n$ matrices
%
\begin{align*}
	M^\mathrm{func}_{ij} &\vcentcolon= \frac{1}{|\ca{E_M}|} \sum_{\ca{M} \cong \ca{H} \leq \ca{G}} \bb{I} \big\{ \{i,j\} \in \ca{A}(\ca{H}) \big\} \sum_{e \in \ca{E_H}} W(e)\,, \\
	M^\mathrm{struc}_{ij} &\vcentcolon= \frac{1}{|\ca{E_M}|} \sum_{\ca{M} \cong \ca{H} < \ca{G}} \bb{I} \big\{ \{i,j\} \in \ca{A}(\ca{H}) \big\} \sum_{e \in \ca{E_H}} W(e)\,.
\end{align*}
\end{definition}


\begin{remark}
Example~\ref{ex:motif_adj_matrices} gives a simple illustration of calculating an MAM.
When $W \equiv 1$ and $\ca{M}$ is simple, the (functional or structural) MAM entry $M_{ij} \ (i \neq j)$ simply counts the (functional or structural) instances of $\ca{M}$ in $\ca{G}$ containing $i$ and $j$.
When $\ca{M}$ is not simple, $M_{ij}$ counts only those instances with anchor sets containing both $i$ and $j$.
MAMs are always symmetric, since the only dependency on $(i,j)$ is via the unordered set $\{i,j\}$.
\end{remark}









\subsection{Computation} \label{sec:graphs_computation}

In order to state Propositions \ref{prop:motif_adj_matrix_formula} and~\ref{prop:motif_adj_matrix_computation}, we need one more definition.

\begin{definition}[Anchored automorphism classes]
Let $(\ca{M,A})$ be a motif.
Let $S_\ca{M}$ be the set of permutations on $ \ca{V_M} = \{ 1, \ldots, m \}$ and define the \emph{anchor-preserving permutations} $S_\ca{M,A} = \{ \sigma \in S_\ca{M} : \{1,m\} \subseteq \sigma(\ca{A}) \}$.
Let $\sim$ be the equivalence relation defined on $S_\ca{M,A}$ by: $\sigma \sim \tau \iff \tau^{-1} \sigma$ is an automorphism of $\ca{M}$.
Finally the \emph{anchored automorphism classes} are the quotient set $S_\ca{M,A}^\sim \vcentcolon= S_\ca{M,A} \ \big/ \sim$\,.
\end{definition}


\begin{proposition}[MAM formula] \label{prop:motif_adj_matrix_formula}
Let $\ca{G} = (\ca{V,E},W)$ be a graph with vertex set ${\ca{V}=\{1,\ldots,n\}}$ and let $(\ca{M,A})$ be a motif on $m$ vertices. Then for any $i,j \in \ca{V}$ and with $k_1 = i$, $k_m = j$, the functional and structural MAMs of $\ca{(M,A)}$ in $\ca{G}$ are given by
%
%
\begin{align*}
M^\mathrm{func}_{ij} &= \frac{1}{|\ca{E_M}|} \sum_{\sigma \in S_\ca{M,A}^\sim} \ \sum_{\{k_2, \ldots, k_{m-1}\} \subseteq \ca{V}} \ J^\mathrm{func}_{\mathbf{k},\sigma} \ G^\mathrm{func}_{\mathbf{k},\sigma}\,, &(1) \\
M^\mathrm{struc}_{ij} &= \frac{1}{|\ca{E_M}|} \sum_{\sigma \in S_\ca{M,A}^\sim} \ \sum_{\{k_2, \ldots, k_{m-1}\} \subseteq \ca{V}} \ J^\mathrm{struc}_{\mathbf{k},\sigma} \ G^\mathrm{struc}_{\mathbf{k},\sigma}\,, &(2) 
\end{align*}
%
where
%
\begin{align*}
	\ca{E}_\ca{M}^0 &\vcentcolon= \{ (u,v) : 1 \leq u < v \leq m : (u,v) \notin \ca{E_M}, (v,u) \notin \ca{E_M} \}\,, \\
	\ca{E}_\ca{M}^\mathrm{s} &\vcentcolon= \{ (u,v) : 1 \leq u < v \leq m : (u,v) \in \ca{E_M}, (v,u) \notin \ca{E_M} \}\,, \\
	\ca{E}_\ca{M}^\mathrm{d} &\vcentcolon= \{ (u,v) : 1 \leq u < v \leq m : (u,v) \in \ca{E_M}, (v,u) \in \ca{E_M} \}\,,
\end{align*}
%
are respectively the missing edges, single edges and double edges of $\ca{E_M}$, and
%
%TC:ignore
\begin{alignat*}{3}
%
	J^\mathrm{func}_{\mathbf{k},\sigma}
	& \vcentcolon= \prod_{\ca{E}_\ca{M}^0} (J_\mathrm{n})_{k_{\sigma u},k_{\sigma v}}
	&& && \prod_{\ca{E}_\ca{M}^\mathrm{s}} J_{k_{\sigma u},k_{\sigma v}}
	\prod_{\ca{E}_\ca{M}^\mathrm{d}} (J_\mathrm{d})_{k_{\sigma u},k_{\sigma v}}\,, \\
%
	G^\mathrm{func}_{\mathbf{k},\sigma}
	& \vcentcolon= \sum_{\ca{E}_\ca{M}^\mathrm{s}} G_{k_{\sigma u},k_{\sigma v}}
	&& + && \sum_{\ca{E}_\ca{M}^\mathrm{d}} (G_\mathrm{d})_{k_{\sigma u},k_{\sigma v}}\,, \\
%
	J^\mathrm{struc}_{\mathbf{k},\sigma}
	& \vcentcolon= \prod_{\ca{E}_\ca{M}^0} (J_0)_{k_{\sigma u},k_{\sigma v}}
	&& && \prod_{\ca{E}_\ca{M}^\mathrm{s}} (J_\mathrm{s})_{k_{\sigma u},k_{\sigma v}}
	\prod_{\ca{E}_\ca{M}^\mathrm{d}} (J_\mathrm{d})_{k_{\sigma u},k_{\sigma v}}\,, \\
%
	G^\mathrm{struc}_{\mathbf{k},\sigma}
	&\vcentcolon= \sum_{\ca{E}_\ca{M}^\mathrm{s}} (G_\mathrm{s})_{k_{\sigma u},k_{\sigma v}}
	&& + && \sum_{\ca{E}_\ca{M}^\mathrm{d}} (G_\mathrm{d})_{k_{\sigma u},k_{\sigma v}}\,. 
%
\end{alignat*}
%TC:endignore
\end{proposition}
%
\begin{proof}
See Proof~\ref{proof:motif_adj_matrix_formula}.
\end{proof}




\begin{proposition}[Complexity of MAM formula] \label{prop:motif_adj_matrix_computation}
Suppose that ${m \leq 3}$, and the adjacency matrix $G$ of $\ca{G}$ is known.
Then computing adjacency and indicator matrices and calculating an MAM using Equations $(1)$ and $(2)$ in Proposition~\ref{prop:motif_adj_matrix_formula} involves at most 18 matrix multiplications, 22 entry-wise multiplications and 21 additions of (typically sparse) $n \times n$ matrices.
\end{proposition}

\begin{proof}
See Proof~\ref{proof:motif_adj_matrix_computation}.
\end{proof}


Hence for motifs on at most three vertices and with sparse adjacency matrices, Proposition~\ref{prop:motif_adj_matrix_formula} gives a fast and parallelisable matrix-based procedure for computing MAMs. In practice, additional symmetries of the motif often allow computation with even fewer matrix operations, demonstrated in Example~\ref{ex:motif_adj_calc}.

A list of such MAM formulae for all simple motifs on at most three vertices (up to isomorphism), as well as for the \emph{collider} and \emph{expander} motifs (Section~\ref{sec:coll_expa}), is given in Table~\ref{tab:motif_adj_mat_table}. These formulae are generalisations of those stated in Table S6 in the supplementary materials for \cite{benson2016higher}, in an incomplete list of only \emph{structural} MAMs of \emph{unweighted} graphs. Note that the functional MAM formula for the two-vertex motif $\ca{M}_\mathrm{s}$ yields the symmetrised adjacency matrix $M = G + G^\top$ which is used for traditional spectral clustering (Section~\ref{sec:spectral_overview}). The question of whether to use functional or structural MAMs for motif-based spectral clustering will be addressed in Section~\ref{sec:spectral_motifrwspectclust}.

